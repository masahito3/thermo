% Options for packages loaded elsewhere
\PassOptionsToPackage{unicode}{hyperref}
\PassOptionsToPackage{hyphens}{url}
%
\documentclass[
]{ltjsarticle}
\usepackage{amsmath,amssymb}
\usepackage{setspace}
\usepackage{iftex}
\ifPDFTeX
  \usepackage[T1]{fontenc}
  \usepackage[utf8]{inputenc}
  \usepackage{textcomp} % provide euro and other symbols
\else % if luatex or xetex
  \usepackage{unicode-math} % this also loads fontspec
  \defaultfontfeatures{Scale=MatchLowercase}
  \defaultfontfeatures[\rmfamily]{Ligatures=TeX,Scale=1}
\fi
\usepackage{lmodern}
\ifPDFTeX\else
  % xetex/luatex font selection
  \ifLuaTeX
    \usepackage[ipaex]{luatexja-preset}
  \fi
\fi
% Use upquote if available, for straight quotes in verbatim environments
\IfFileExists{upquote.sty}{\usepackage{upquote}}{}
\IfFileExists{microtype.sty}{% use microtype if available
  \usepackage[]{microtype}
  \UseMicrotypeSet[protrusion]{basicmath} % disable protrusion for tt fonts
}{}
\makeatletter
\@ifundefined{KOMAClassName}{% if non-KOMA class
  \IfFileExists{parskip.sty}{%
    \usepackage{parskip}
  }{% else
    \setlength{\parindent}{0pt}
    \setlength{\parskip}{6pt plus 2pt minus 1pt}}
}{% if KOMA class
  \KOMAoptions{parskip=half}}
\makeatother
\usepackage{xcolor}
\usepackage[a4paper,margin=0.5in]{geometry}
\setlength{\emergencystretch}{3em} % prevent overfull lines
\providecommand{\tightlist}{%
  \setlength{\itemsep}{0pt}\setlength{\parskip}{0pt}}
\setcounter{secnumdepth}{-\maxdimen} % remove section numbering
\usepackage{graphicx}
\graphicspath{{images/}}
\usepackage{etoolbox}
\gappto{\normalsize}{%
\abovedisplayskip=0em%
\belowdisplayskip=0em%
\abovedisplayshortskip=0pt%
\belowdisplayshortskip=0pt}
\gappto{\footnotesize}{%
\abovedisplayskip=0em%
\belowdisplayskip=0em%
\abovedisplayshortskip=0pt%
\belowdisplayshortskip=0pt}
\setlength{\parskip}{0pt}
\setlength{\parindent}{0pt}
\providecommand{\l}{}
\renewcommand{\l}{\left}
\providecommand{\r}{}
\renewcommand{\r}{\right}
\newcommand{\so}{\text{{\large\therefore\ }}}
\newcommand{\red}[1]{\textcolor{red}{#1}}
\newcommand{\kome}[1]{\red{(*#1)}}
\newcommand{\disp}{\displaystyle}
\newcommand{\iif}{\rightleftarrows}
\newcommand{\cuz}{\text{{\large\because\ }}}
\newcommand{\eps}{\epsilon}
\newcommand{\D}{\Delta}
\newcommand{\limto}[2]{\lim_{#1\to #2}}
\newcommand{\sumto}[2]{\sum_{#1}^{#2}}
\newcommand{\V}[1]{\vec{#1}}
\allowdisplaybreaks
\providecommand{\series}{}
\renewcommand{\series}[1]{\sumto{#1=0}{\infty}}
\renewcommand{\linespread}{2.0}
\ifLuaTeX
  \usepackage{selnolig}  % disable illegal ligatures
\fi
\IfFileExists{bookmark.sty}{\usepackage{bookmark}}{\usepackage{hyperref}}
\IfFileExists{xurl.sty}{\usepackage{xurl}}{} % add URL line breaks if available
\urlstyle{same}
\hypersetup{
  hidelinks,
  pdfcreator={LaTeX via pandoc}}

\author{}
\date{}

\begin{document}

{
\setcounter{tocdepth}{2}
\tableofcontents
}
\setstretch{1.5}
\newpage

\hypertarget{p.12-ux88dcux8db3-ux3079ux304dux7d1aux6570ux306eux5408ux6210-25-6.1}{%
\subsection{P.12 補足 べき級数の合成 \textquotesingle25
6.1}\label{p.12-ux88dcux8db3-ux3079ux304dux7d1aux6570ux306eux5408ux6210-25-6.1}}

\(|x-a|<R_f\) ならば
\(f(x)=\displaystyle\sum_{n=0}^{\infty} a_n(x-a)^n\) とする

\(|x-b|<R_g\) ならば
\(g(x)=\displaystyle\sum_{m=0}^{\infty} b_m(x-a)^m\) とする

\(R_f\), \(R_g\) は収束半径とする。このとき

\(|x-b|<R_g\) かつ \(\displaystyle\sum_{m=0}^{\infty}|c_m(x-b)^m|<R_f\),
\(c_m=\displaystyle\renewcommand{\arraystretch}{0.8}\begin{cases}b_0-a & (m=0) \\ b_m & (m>0) \end{cases}\)
ならば

\(c_m=\Bigg\{\begin{array}{ll}b_0-a & (m=0) \rule[-8pt]{0pt}{20pt} \\ b_m & (m>0) \end{array}\)
\(c_m=\bigg\{\renewcommand{\arraystretch}{0.6}\begin{array}{ll}b_0-a & (m=0) \rule[-4pt]{0pt}{10pt} \\ b_m & (m>0) \end{array}\)
\(c_m=\Bigg\{\begin{array}{ll}b_0-a & (m=0) \rule[-4pt]{0pt}{10pt} \\ b_m & (m>0) \end{array}\)

\(f(g(x))\) は \(b\) を中心としてべき級数であらわされる

\noindent\rule{\linewidth}{0.4pt}

(証明)

\(|x-b|<R_g\) とする

\begin{flalign*}
&\begin{array}{ll}AAA & BBB \\ CCC & ddd \\ \end{array} &\\
&AAABBBCCC&
\end{flalign*}

上

上 \begin{flalign*}
&AAA = BBB &\\
&CCC CCC &
\end{flalign*} 中 \begin{flalign*}
AAA &= BBB &\\
    &= CCC &
\end{flalign*} 下

\end{document}
